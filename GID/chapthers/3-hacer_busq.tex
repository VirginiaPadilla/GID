\setchapterstyle{kao}

\setchapterimage[6cm]{gid-3-org}
\setchapterpreamble[u]{\margintoc}

\chapter{Organización de información documental }
\label{ch:org-info}
\index{informaci\'on documental!organizaci\'on}

 Una vez que se recupera la información , se procede a valorarla; es decir, a analizar y evaluar su relevancia de acuerdo con el problema y la hipótesis de investigación. De cada uno de los documentos seleccionados, se conservan sus respectivas fichas bibliográficas referenciales. Toda esta información es recopilada con el gestor documental Zotero.
 
 \marginnote [1cm] {
 	\begin{kaobox}[frametitle= Zotero]
 		Zotero es una herramienta gratuita y fácil de usar que le ayuda a recopilar, organizar, citar y compartir investigaciones. Los siguientes son enlaces de ayuda:\\
 		\href{https://www.zotero.org/}{P\'agina de Zotero} 
 		\href{https://www.youtube.com/watch?v=7iz0lJLI5L4}{Zotero y Word}  y
 		\href{https://www.zotero.org/support/es/quick_start_guide}{Tutorial}
 	\end{kaobox}
 } 
 
 	\begin{marginfigure}[7cm]%
 		\includegraphics[width=\linewidth]{imagen2}
 	%	\caption{Fichas bibliogr\'aficas }
 	%	\label{fig:Fichas}
 	\end{marginfigure}
 
 	
 	
 
Las fichas bibliográficas contienen información acerca de:
 Datos relacionados con el tipo de publicación, ubicación e identificación del documento y las respectivas categorías de análisis. Se establece si es un artículo de revista, libro producto de investigación, capítulo de libro, trabajo de investigación, sistema de investigación, o evento (ponencia); además, si es una   investigación que forma parte de la educación formal como monografía, trabajo de investigación de maestría o tesis doctoral. 
 
 
 La ubicación del documento se refiere al lugar donde se encontró, esto es, en bases de datos, bibliotecas, archivo personal o un recurso Web; además, se aclara si el documento es en texto completo o referencial; el nombre del sistema en que se recuperó la información; si está en formato impreso o digital; la dirección electrónica y la fecha de consulta. 
 
 
 La identificación del documento se realiza a partir del título, del autor, del título de la publicación, del país (bien sea de la publicación o del autor, aquí no se discriminó), del idioma en que se encuentra el texto, del número de páginas y de las palabras clave que presenta la publicación.
 
 

\section{Criterios de selección de documentos}

 \index{documentos!criterios de selecci\'on}     
Los criterios de selección se encuentran determinados por los objetivos de la revisión, es decir la pregunta a la que trata de responder el artículo. Otro de los aspectos que determina la selección de los artículos es su calidad metodológica y si cumplen con los criterios de calidad científica buscada. En una primera fase los aspectos que deberemos tener en cuenta serán:  \textit{el título, los autores, el resumen} y\textit{ los resultados}.


\begin{marginfigure}[-3,2cm]%
	\includegraphics[width=\linewidth]{criterio}
	%	\caption{Fichas bibliogr\'aficas }
	%	\label{fig:Fichas}
\end{marginfigure}

Otros criterios que se podrían tener en cuenta son:  \textit{tipo de publicación , el título, los autores, el resumen} y \textit{los resultados, cantidad de citaciones}

Luego, se procede al análisis respectivo mediante una lectura crítica. La pagina \href{https://www.redcaspe.org/}{CASPe}  define la lectura crítica como una técnica que ofrece la oportunidad de aumentar la efectividad de nuestra lectura, adquiriendo las habilidades necesarias para excluir con la mayor prontitud los artículos científicos de mala calidad y aceptar aquellos otros con la suficiente calidad científica para ayudarnos en nuestra toma de decisiones para el cuidado de los pacientes. 

\begin{marginfigure}[-3.2cm]%
	\includegraphics[width=\linewidth]{criterio2}
	%	\caption{Fichas bibliogr\'aficas }
	%	\label{fig:Fichas}
\end{marginfigure}

\begin{marginfigure}[3.2cm]%
	\includegraphics[width=\linewidth]{imagen1}
	%	\caption{Fichas bibliogr\'aficas }
	%	\label{fig:Fichas}
\end{marginfigure}



Proponen que los artículos científicos deben ser evaluados en tres aspectos: 


\begin{itemize}
	\item ?`Podemos confiar en los resultados? Dicho de otra forma:  ?`son válidos? Es decir,   enjuiciamos la validez metodológica del artículo.  Dependiendo de la validez de un artículo lo podemos clasificar dentro de una escala de niveles de evidencia y grados de recomendación. 

 	\item   ?`Cuáles son los resultados? Por ejemplo, ?`Sistema de información documental frente a Sistemas didáctico muestra alguna relación ?  ?`cómo miden esta relación ? ?`son precisos los resultados?

	\item ?`Son pertinentes o aplicables estos resultados en mi área ?
\end{itemize}


\begin{marginfigure}[1.2cm]%
	\includegraphics[width=\linewidth]{pregunta}
	%	\caption{Fichas bibliogr\'aficas }
	%	\label{fig:Fichas}
\end{marginfigure}

\begin{kaobox}[frametitle=Ejercicio]
	Establezca los criterios que va a usar para la revisi\'on de sus documentos. %%%%%
\end{kaobox}


%%%%%%%%%%%%%%%%
\section{Mapas para organizar documentos}
 \index{documentos!mapas para organizar}     

 No hay una organización establecida para los art\'iculos de investigaci\'on provenientes de una revisión documental. Por tanto, cada autor tendrá que elaborar la suya propia. La
regla fundamental para escribir un trabajo de esta clase es preparar un guión \cite{GuiraoGoris2008}.


\begin{marginfigure}[1cm]%
	\includegraphics[width=\linewidth]{mapaconceptual}
	%	\caption{Fichas bibliogr\'aficas }
	%	\label{fig:Fichas}
\end{marginfigure}
 Si la información esta bien organizada, cuenta con una estructura lógica que va introduciendo de forma secuencial y razonable la información, el artículo estará mejor redactado y permitirá una fácil lectura y comprensión. Por tanto, la elaboración de un guión es fundamental.
 
De acuerdo a \GU, la  metodología para la elaboración
de un guión tras una lectura crítica de los textos se basa en los siguientes pasos:  Ordenar, Rotular, Integrar, Priorizar. 


\begin{marginfigure}[2.2cm]%
	\includegraphics[width=\linewidth]{revision5}
	%	\caption{Fichas bibliogr\'aficas }
	%	\label{fig:Fichas}
\end{marginfigure}

\begin{description}
	\item[Ordenar:] Se reduce la información eliminando todo aquello que no es esencial mediante un proceso que pasa por segmentar la información básica. Ordenar dicha información por conjuntos, agrupando esta información esta comienza a adquirir  características comunes.
  
	\item[Rotular:] En esta fase se asigna un nombre a cada grupo. Si no es posible decidir el nombre, se puede asignar un código arbritario como una letra, color o número.
	
	\item[Integrar:]  Con los grupos ya formalizados y
	etiquetados se procede a integrar los grupos que se parezcan bastante, de forma que habrá algunos que queden aislados y otros integrados.  
	
	\item[Priorizar:] Tras estos procesos se priorizan los grupos para identificar la información que será más relevante dentro de la organización alcanzada.
\end{description}


\begin{marginfigure}[1cm]%
	\includegraphics[width=\linewidth]{mapamental2}
	%	\caption{Fichas bibliogr\'aficas }
	%	\label{fig:Fichas}
\end{marginfigure}
 
  
En este proceso de estructuración de los datos se puede recurrir a la realización de un mapa conceptual o un mapa mental según sea lo más apropiado. Se muestra la estructura de la informaci\'on en un mapa mental que se usa en el documento \GU ,  en la figura \ref{fig:mapa}

 


\begin{figure}%
	\includegraphics {MapaMental}
	\caption{Mapa Mental. Tomado de \GU }
	\label{fig:mapa}
\end{figure}

A medida que se localizan documentos, la información se va incluyendo en este esquema o percha de análisis de modo que tras haber realizado toda la lectura de la bibliografía y seleccionado la información más relevante se cuelga de cada percha o ítem. De esta manera se puede ir combinando la información de diferentes fuentes en una estructura de carácter común para iniciar la redacci\'on del artículo de revisi\'on. 
En la figura \ref{fig:percha} se ejemplifica cómo la información extraída de diferentes fuentes se organizó en cada ítem de la percha en el \GU.




\begin{figure}%
	\includegraphics[width=0.8\linewidth]{Percha}
		\caption{Percha de an\'alisis. Tomado de \GU }
		\label{fig:percha}
\end{figure}
 


%%%%


 \section{Resultados de la búsqueda}
 
  \index{documentos!resultados de la b\'usqueda}     
 
 Los resultados de la búsqueda y revisión, se puede presentar en el cuadro \ref{tab:resultados} donde por cada recurso se plasman los resultados parciales que representan la cantidad de documentos arrojados por la búsqueda. Y en resultados definitivos se colocan la cantidad de artículos que se seleccionaron una vez que se aplicó  el criterio de selección de documentos.
 
 \begin{marginfigure}[-3.2cm]%
 	\includegraphics[width=\linewidth]{imagen3}
 	%	\caption{Fichas bibliogr\'aficas }
 	%	\label{fig:Fichas}
 \end{marginfigure}
 


\begin{table}[h]\index{{recursos!resultados}}
	\footnotesize%
	\begin{center}
		\footnotesize
		\begin{tabular}{|c|c|c|}
			\hline
			 Recurso & Resultados Parciales & Resultados Definitivos  \\
			\hline  			
		 {LISTA} &   &  \\
		 \hline
		  {LISA} &   &  \\
		  \hline
		  Google Scholar &   &  \\
		  \hline
	         	  &   &  \\
	         	  \hline
	              &   &  \\
	              
			\hline  			
			
		\end{tabular}
	\end{center}
	\caption{Resultados por Recursos }
	\label{tab:resultados}
\end{table}


\begin{marginfigure}[-1.2cm]%
	\includegraphics[width=\linewidth]{pregunta}
	%	\caption{Fichas bibliogr\'aficas }
	%	\label{fig:Fichas}
\end{marginfigure}

\begin{kaobox}[frametitle=Ejercicio]

	 Una vez obtenido los resultados de su búsqueda en las bases de datos, coloque la cantidad de documentos arrojados por la búsqueda en  la columna\textit{ Resultados Parciales} de la tabla Resultados por Recursos.
	Note que puede ajustar o restringir este número, perfeccionando los criterios de búsqueda.
	Defina cu\'ales son los criterios de evaluación de documento que va a utilizar y aplíquelo a sus resultados parciales. Una vez finalizado, plasme en la columna \textit{Resultados Definitivos} de la ya mencionada tabla, la cantidad de documentos elegidos.
	Indique:
		\begin{itemize}
			\item ?`Que criterios de evaluación utilizó?
			\item  Llene la Tabla de Resultados por Recursos con sus resultados obtenidos.
			\item Presente mapa mental o conceptual con la  informaci\'on recolectada.
		\end{itemize}
	\end{kaobox}


\section{Bibliograf\'ia de la b\'usqueda}

 \index{documentos!bibliograf\'ia}     

El resultados de la selección de artículos conforma la bibliografía de su informe. Esta debe estar construida de acuerdo a las normas APA \sidecite{Libertador2011}  Al usar gestor de referencias bibliográficas  como Zotero, esta bibliografía se tiene disponible en este sistema normas. 



\begin{table}[h] 
	\footnotesize%
	\begin{center}
		\footnotesize
		\begin{tabular}{|c|c|}
			\hline
			\'Indice (autor (es), fecha) & Descripci\'on del documento     \\
			\hline  			
			  &    \\
			  \hline
			    &  \\
			    \hline
			   &  \\
			   \hline
		  &  \\
		 
			\hline  			
			
		\end{tabular}
	\end{center}
	\caption{Bibliograf\'ia }
	\label{tab:biblio}
\end{table} 


\begin{marginfigure}[1.2cm]%
	\includegraphics[width=\linewidth]{pregunta}
	%	\caption{Fichas bibliogr\'aficas }
	%	\label{fig:Fichas}
\end{marginfigure}

\begin{kaobox}[frametitle=Ejercicio]
	Plasme a continuación la bibliografía con sus artículos seleccionados y preséntela en la tabla \ref{tab:biblio}:
\end{kaobox}

 