\setchapterpreamble[u]{\margintoc}
\setchapterimage{gid-3-org}
%\setchapterpreamble[u]{\margintoc}

\chapter*{Introducci\'on}
 
\labch{intro}

%%\section{The Main Ideas}

El objetivo de este  documento es que sirva de gu\'ia de apoyo para el  curso Gestión de Información Documental, asignatura introductoria  del Postgrado de Finanzas de la Universidad Nacional Experimental de Guayana. 

El propósito de la asignatura Gesti\'on de la Informaci\'on Documental es que el participante conozca cómo  formular estrategias de búsqueda y recuperación de información, evaluar la confiabilidad de las fuentes de información disponibles y construir una base de datos personal con el apoyo de un gestor bibliográfico que permita su registro sistemático y comunicación en el\href{https://normasapa.com/normas-apa-2019-cuestiones-mas-frecuentes/comment-page-11/}{ estilo APA}. 

La asignatura se dicta a trav\'es  del aula virtual \href{http://moodle.uneg.edu.ve/moodle/}{Gesti\'on de Informaci\'on Documental} construida bajo la plataforma de aprendizaje en l\'inea  \href{https://moodle.org/?lang=es}{Moodle} de la Universidad Nacional Experimental de Guayana.



El presente documento esta estructurado en cuatro cap\'itulo que corresponde con las secciones que conforman el aula virtual del curso. A saber:
\begin{description}
	\item [Revisi\'on Documental.] Se define la Revisi\'on Documental y se exponen los tipos de revisiones que se pueden elaborar.

	\item [Metodolog\'ia de la Revisi\'on Documental.] La metodolog\'ia para la revisi\'on documental se inicia en esta sesi\'on con la construcci\'on de las cadenas de b\'usqueda de informaci\'on; adem\'as de, la escogencia de la fuentes de informaci\'on.

	\item [Organizaci\'on Informaci\'on Documental.] En esta sesi\'on se presentan las t\'ecnicas y criterios para organizar y seleccionar los documentos relevantes a la revisi\'on que se va a hacer.

	\item [Redacci\'on del Documento de Revisi\'on.] Se presentan los criterios para una redacci\'on fluida y clara.
	
\end{description}


Las figuras de las portadas y  encabezados de los cap\'itulos fueron elaborados por Andrea Isabel Revilla. Pueden ver su portafolio  en Instagram:  @ave.rebel \\ Twitter: @AveRebel \\ ArtStation \href{https://linktr.ee/avebel}{avebel} . Gracias Andrea Isabel! 

Este documento fue escrito con la ayuda de \href{https://sourceforge.net/projects/koma-script/}{\KOMAScript}  \href{https://www.latex-project.org/}{\LaTeX} usando la clase \href{https://github.com/fmarotta/kaobook/}{kaobook}.

Para la redacci\'on del documento se usan las normas de estilo de la \href{https://uc3m.libguides.com/guias_tematicas/citas_bibliograficas/estilo-ieee}{IEEE}.

Cualquier error u omisi\'on es de mi absoluta responsabilidad. Agradezco que env\'ien su  comentarios, observaciones o quejas a la direcci\'on electr\'onica virginiapadillas@gmail.com.



