\setchapterstyle{kao}

\setchapterimage[6cm]{gid-3-org}
\setchapterpreamble[u]{\margintoc}

\chapter{Redacción  del documento de revisión }
\label{ch:org-info}
\index{revisi\'on documental!redacci\'on de documentos}



 Redactar, etimológicamente significa compilar o poner en orden; en un sentido más preciso consiste en expresar por escrito los pensamientos o conocimientos ordenados con anterioridad. Para redactar un artículo de carácter científico  en el \GU  se  apunta las cualidades esenciales de un buen estilo: 
 
 \begin{marginfigure}[-1.2cm]%
 	\includegraphics[width=\linewidth]{revision4}
 	%	\caption{Fichas bibliogr\'aficas }
 	%	\label{fig:Fichas}
 \end{marginfigure}
 
 
 
 \begin{description}
 	\item [Claridad.] Se es claro cuando el escrito penetra sin esfuerzo en la mente del lector. Para lograr la claridad no basta tener las ideas claras. Es necesario que la construcción de la frase y el párrafo responda al orden lógico de las ideas. Para asegurar esto último es conveniente ligar las ideas entre dos o más frases. 
 	
 	\item[Concisión.] se es conciso cuando se usa sólo las palabras indispensables, precisas y significativas para expresar lo que se quiere decir. Ello implica brevedad, centrando el mensaje en lo esencial. Conciso no quiere decir lacónico sino denso. Lo contrario es la vaguedad, la imprecisión y el exceso de palabras. 
 	
 	\item[Precisión] Se es preciso cuando se usa un lenguaje sin términos ambiguos ni expresiones confusas o equívocas. Precisión supone exactitud. 
 	
 	\item[Sencillez y naturalidad.] Está presente cuando usamos lenguaje común sin caer en la vulgaridad. La sencillez supone huir de lo enrevesado, lo artificioso, lo barroco y de lo complicado.
 	
 \end{description}

 \begin{marginfigure}[-2.2cm]%
	\includegraphics[width=\linewidth]{revision5}
	%	\caption{Fichas bibliogr\'aficas }
	%	\label{fig:Fichas}
\end{marginfigure}



El elemento central de una buena redacción se encuentra en suministrar la información siguiendo un proceso lógico y paulatino de forma que primero redactamos las ideas que son antecedentes y con posterioridad se desarrollan las ideas consecuentes.

\section{Estructura del artículo de revisión}


 \index{revisi\'on documental!estructura de documentos}
Como esquema general de una revisión se recomienda que tenga una breve «introducción», donde se debería plantear la necesidad de abordar la pregunta o preguntas que queremos contestar (el tema a revisar); un apartado sobre «metodología», en el que se exponga cómo, con qué criterios y qué trabajos se han seleccionado y revisado; un apartado de «desarrollo y discusión», en el que se presentan los detalles más destacables de los artículos revisados (diseños, sesgos, resultados, etc.) y la síntesis discutida y argumentada de los resultados.

 \begin{marginfigure}[1.2cm]%
	\includegraphics[width=\linewidth]{analisis}
	%	\caption{Fichas bibliogr\'aficas }
	%	\label{fig:Fichas}
\end{marginfigure}

 

En la sección «conclusión» se presentan las consecuencias que extraemos de la revisión, propuestas de nuevas hipótesis y líneas de investigación concretas para el futuro.  Un esquema de la estructura se describe en la  tabla 


\begin{table}[h] 
	\footnotesize%
	\begin{center}
		\footnotesize
		\begin{tabular}{|c|c|}
		\hline
			\'Introducci\'n   & Definir objetivos     \\
		\hline  			
		    Metodología	&  Búsqueda bibliográfica. \\
		    	& Criterios de selección. \\
		    	&  Recuperación de la información. \\
		  		& Evaluación de  artículos seleccionados. \\
		  		& Criterios de selección. \\
		  		& Análisis  de fiabilidad y validez de los artículos \\  
		 \hline
			Desarrollo &  Organización y estructuración de los datos. \\
			 
		    & Combinación de los resultados de diferentes originales. \\
			& Argumentación crítica de los resultados   \\
		\hline
		   Conclusión	&  \\
		\hline
			Bibliografía	&  \\
			
		\hline  			
			
		\end{tabular}
	\end{center}
	\caption{Bibliograf\'ia }
	\label{tab:biblio}
\end{table} 

 