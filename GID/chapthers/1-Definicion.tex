
\setchapterimage[6cm]{gid-3-org}
\setchapterpreamble[u]{\margintoc}

\chapter{Revisi\'on Documental}
\label{ch:def-RD}


\textbf{ ?`Qu\'e es la Revisi\'on Documental?}

\index{revisi\'on documental}

En el documento  de \GU definen la revisión de documentos como: \textit{ver con atención y cuidado o someter algo a nuevo examen para corregirlo, enmendarlo o repararlo}. El documento de revisión, llamado también el estado del arte o revisión bibliográfica,  es considerado como un estudio detallado, selectivo y crítico que integra la información esencial en una perspectiva unitaria y de conjunto \sidecite{Naranjo2010}. \index{revisi\'on documental!definici\'on}


 Un artículo de revisión no es una publicación original y su finalidad es examinar la bibliografía publicada y situarla en cierta perspectiva. La revisión se puede reconocer como un estudio en sí mismo, en el cual el revisor tiene un interrogante, recoge datos (en la forma de artículos previos), los analiza y extrae una conclusión.
 
 \begin{marginfigure}[-1cm]%
 	\includegraphics[width=\linewidth]{imagen5}
 	%	\caption{Fichas bibliogr\'aficas }
 	%	\label{fig:Fichas}
 \end{marginfigure}



 
 La diferencia fundamental entre una revisión y un trabajo original o estudio primario, es la unidad de análisis, no los principios científicos que se aplican. El objetivo fundamental del artículo de revisión intenta identificar qué se conoce del tema, qué se ha investigado y qué aspectos permanecen desconocidos.
 
 Otros propósitos de la revisión documental es, según  \sidecite{GuiraoGoris2008}:     \index{revisi\'on documental!prop\'osito}
 \begin{itemize}
 	\item   Resumir información sobre un tema o problema. 
 	\item   Identificar los aspectos relevantes conocidos, los desconocidos y los controvertidos sobre el tema revisado. 
	\item   Identificar las aproximaciones teóricas elaboradas sobre el tema.
	\item 	Conocer las aproximaciones metodológicas al estudio del tema.
	\item 	Identificar las variables asociadas al estudio del tema.
	\item 	Proporcionar información amplia sobre un tema.
	\item 	Ahorrar tiempo y esfuerzo en la lectura de documentos primarios.
	\item 	Ayudar al lector a preparar comunicaciones, clases, protocolos.
	\item 	Contribuir a superar las barreras idiomáticas.
	\item 	Discutir críticamente conclusiones contradictorias procedentes de diferentes estudio.
	\item 	Mostrar la evidencia disponible.
	\item 	Dar respuestas a nuevas preguntas. 
	\item 	Sugerir aspectos o temas de investigación 
 \end{itemize}

 \begin{marginfigure}[-5cm]%
	\includegraphics[width=\linewidth]{imagen4}
	%	\caption{Fichas bibliogr\'aficas }
	%	\label{fig:Fichas}
\end{marginfigure}
 
\section{Tipos de Revisi\'on Documental}
\index{revisi\'on documental!tipos}

\begin{description}
	\item[La revisión narrativa o cualitativa.]  En las que tras seleccionar un número determinado de artículos, se agrupan  por el sentido de sus resultados y se discuten a la luz de las características metodológicas de cada estudio, para derivar una conclusión más o menos general tras su examen. Esta revisión narrativa incluye una cuantificación simple  de los resultados, donde se limita a contabilizar el número de estudios con resultados positivos y negativos. Este sistema de revisión «por votos» a favor y en contra  ofrece realmente poca información al lector. \index{revisi\'on documental!narrativa}
	
	\begin{marginfigure}[-3cm]%
		\includegraphics[width=\linewidth]{revision1}
		%	\caption{Fichas bibliogr\'aficas }
		%	\label{fig:Fichas}
	\end{marginfigure}
	
	
	\item[La revisión exhaustiva.]  Se trata de un artículo de bibliografía comentada, son trabajos bastante largos, muy especializados y no ofrecen información precisa a un profesional interesado en responder a una pregunta específica. \index{revisi\'on documental!exhaustiva}
	
	\item[La revisión descriptiva.] Proporciona al lector una puesta al día sobre conceptos útiles en áreas en constante evolución. Este tipo de revisiones tienen una gran utilidad en la enseñanza y también interesará a muchas personas de campos conexos, porque leer buenas revisiones es la mejor forma de estar al día en nuestras esferas generales de interés. \index{revisi\'on documental!descriptiva}
	
	\begin{marginfigure}[-2cm]%
		\includegraphics[width=\linewidth]{imagen2}
		%	\caption{Fichas bibliogr\'aficas }
		%	\label{fig:Fichas}
	\end{marginfigure}
	
	\item[La revisión evaluativa.] Responde a una pregunta específica muy concreta sobre aspectos etiológicos, diagnósticos, clínicos o terapéuticos. Este tipo de revisión son las que actualmente conocemos como preguntas clínicas basadas en la evidencia científica. \index{revisi\'on documental!evaluativa}
	
 
\end{description}
	
	


 


 





